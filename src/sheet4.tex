\documentclass{article}

\usepackage{amsmath}
\usepackage{amssymb}
\usepackage{graphicx}

\begin{document}
\title{Vacation Sheet 4}
\author{Ben Eills}
\maketitle

\textit{Submit neat answers to all the questions you attempt (include all the working you wish to be marked, e.g. calculations, diagrams).  Partial answers should also be handed in: these may be awarded partial credit.}

\begin{enumerate}
	\item We test knowledge of transformations.
		\begin{enumerate}
			\item Graph the following transformations for a constant $a$.
			\begin{enumerate}
				\item $f(x) \longrightarrow f(x + a)$ \\
				
				\item $f(x) \longrightarrow af(x)$ \\
				
				\item $f(x) \longrightarrow f(ax)$ \\
				
				\item $f(x) \longrightarrow f(x) + a$ \\
			\end{enumerate}
			\item Pick some interesting cubic and let this be f(x).  Graph f(x) and each of the above transformations when $a=3$. 
		\end{enumerate}
	
	\item Consider the two points $P=(a, b)$ and $Q=(c, d)$.
		\begin{enumerate}
			\item Suppose that $a \ne c$.  Give the equation of the unique line passing through $P$ and $Q$ in the form $y=mx + c$.  Why is it important that $a \ne c$?

			\item Suppose now that $a = c$.  Give the equation of the line \textit{(you will not be able to write it in the form $y = mx + c$)}.
			
			\item Suppose again that $a \ne c$, and that $P$ and $Q$ lie in the upper-right quadrant (i.e. $a, b, c, d \ge 0$).  Give the area of the triangle bounded by the line passing through $P$ and $Q$ and the coordinate axes.
		\end{enumerate}


	\item The Greek mathematician Euclid set forth certain postulates prefacing his work \textit{The Elements}, the most famous of which was the fifth: \textit{"Given a line L and a point P (not on the line), there is exactly one line passing through P that is parallel to L."}
		\begin{enumerate}
			\item Let $L$ be the line $y= 3x - 7$, and P the point $(1,1)$.  Give the special line passing through P, parallel to L.
			
			\item Consider the (Euclidean) lines $L$ and $M$, given by equations $y=ax + b$ and $y = cx + d$, respectively.  Give conditions on a and b for:
				\begin{enumerate}
					\item $L$ and $M$ to be parallel
					
					\item $L$ and $M$ to be perpendicular
				\end{enumerate}
			
			\item Euclid's fifth postulate is only true of "Euclidean Geometry", that is, geometry in planes.  In spherical geometry (like on the surface of the globe), "lines" are actually "greater circles" going around the whole sphere at its widest point.  Thus the equator and Greenwich Meridian are "lines".  Considering the equator and Paris, how many "lines" can be drawn through Paris that are parallel to the Equator?
			
			
		\end{enumerate}


	
	
	\item We test the differential and integral calculus.
	\begin{enumerate}
	
		\item Suppose $f(x)$ passes through the origin and has derivative $f'(x) = 3x^2 - 1$.  Determine fully $f(x)$.  Using the "difference of squares" trick, factorize it as far as possible.
		
		\item Suppose that the minimum value of $g(x)$ is -1, and that $g(x)$ has derivative $g'(x) = 2x-12$.  Determine fully $g(x)$, factorizing your answer.
		
		\item Differentiate and integrate (the Greek letters are constants):
			\begin{enumerate}
				\item $\alpha x^{\beta}$
				\item $\delta + x^{\dfrac{\sigma}{\phi}}$
			\end{enumerate}
		\item Invent two symbols for your own language, and state the simple differentiation "power rule" in terms of them.  \textbf{Make them sufficiently complex so that they will not exist in any real-world alphabet!}
	\end{enumerate}
	
\end{enumerate}

\end{document}