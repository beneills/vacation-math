\documentclass{article}

\usepackage{amsmath}
\usepackage{amssymb}
\usepackage{graphicx}

\begin{document}
\title{Vacation Sheet 5}
\author{Ben Eills}
\maketitle

\textit{Submit neat answers to all the questions you attempt (include all the working you wish to be marked, e.g. calculations, diagrams).  Partial answers should also be handed in: these may be awarded partial credit.}

\begin{enumerate}
	\item We test knowledge mathematical language.  Decide which of the following are true statements.
		\begin{enumerate}
			\item Graph the following transformations for a constant $a$.
			\begin{enumerate}
				\item $a$ is positive implies that $a^2$ is positive
				
				\item $a^2$ is positive implies that $a$ is positive
				
				\item Whenever $a^2$ is odd, $a+1$ is even
				
				\item The discriminant being positive is a sufficient condition for a quadratic to have two distinct roots.
			\end{enumerate}
		\end{enumerate}
	
	\item Briefly sketch the sine and cosine functions on the interval $[ 0^\circ, 720^\circ ]$.
		\begin{enumerate}
			\item Let $f(x)$ be the sine function.  Sketch on a separate graph the transformation $f(x + 90^\circ)$

			\item Comparing your transformation to the cosine function you sketched, write down a relationship between sine and cosine of the form \\ $sin(x + a) = cos(x)$ for some constant a.
			
			\item If $f(x) = f(x+b)$ for all $x$ and some constant $b$, we say that f has \textbf{period $b$}.  Intuitively, this means that the graph of $f(x)$ "repeats" itself every $b$.  Only very special functions are \textbf{periodic} in this way.  Is sine periodic?  If so, find its period.
			
		\end{enumerate}


	\item We test differentiation and integration technique.  Differentiate and integrate the following.
		\begin{enumerate}
			\item $f(x) = x^{\dfrac{1}{1}} + x^{\dfrac{1}{2}} + \cdots + x^{\dfrac{1}{100}}$
			
			\item $g(x) = x^{\dfrac{1}{\sqrt{2}}}$
			
			\item $h(x) = \dfrac{(x+2)^2}{x^{-\dfrac{1}{2}}}$
			
		\end{enumerate}


	
	
	\item Consider the quadratic $z(x) = x^2 + kx + k$
	\begin{enumerate}
	
		\item Write down the interval for k for which:
			\begin{enumerate}
				\item $z(x)$ has two distinct roots
				\item $z(x)$ has a repeated root
				\item $z(x)$ has no real roots
			\end{enumerate}
		
		\item From now on, suppose that $z(x)$ has two distinct roots.  What is the minimum point of $z(x)$?
		
		\item Does $z(x)$ have a maximum point?
		
		\item We define the \textbf{second derivative} of $z(x)$ to be the derivative of the derivative and write it as $\dfrac{d^2z}{dx^2}$.  Therefore, $\dfrac{d^2z}{dx^2} = \dfrac{d}{dx}{ \dfrac{dz}{dx}}$.  Find the second derivative of $z(x)$.
	\end{enumerate}
	
\end{enumerate}

\end{document}