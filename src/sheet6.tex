\documentclass{article}

\usepackage{amsmath}
\usepackage{amssymb}
\usepackage{graphicx}

\begin{document}
\title{Vacation Sheet 6}
\author{Ben Eills}
\maketitle

\textit{Submit neat answers to all the questions you attempt (include all the working you wish to be marked, e.g. calculations, diagrams).  Partial answers should also be handed in: these may be awarded partial credit.}

\begin{enumerate}
	\item The ancient Babylonians wanted to find a way to compute square roots.  If you think about it, you don't really know how to do this, except in the case of "nice" perfect squares (of fractions involving them).  Suppose for a minute that we're trying to find the square root of 10 (which lies somewhere between 3 and 4), and that we have a first "approximation" $x_0$ that we would like to improve.  If the approximation is too low, then $\dfrac{10}{x_0}$ is too high, but their average $x_1 = \dfrac{x_0 + \frac{10}{x_0}}{2}$ is closer to the true square root i.e. a better approximation.  The Babylonians used this repeatedly to achieve a close estimate to the true root.\\
	
		We define the sequence $(x_n)$ as follows (using 2 as our first guess): \\
		$$x_0 = 2$$
		$$x_{n+1} = \dfrac{x_n + \frac{10}{x_n}}{2}$$
	
		\begin{enumerate}
			\item Our first approximation ($x_0 = 2$) is not very good.  Find the second and third approximations, $x_1$ and $x_2$ (without using a calculator).
			
			\item Which two integers does $\sqrt{1000}$ lie between?
			
			\item Define a sequence $(b_n)$, similar to the above, to find $\sqrt{1000}$ using 20 as your first approximation.
				
			\item Write down $b_0$, $b_1$, and $b_2$ (you may use a calculator for this!)
				
			\item Find $(b_2)^2$ to determine how good your approximation is.  How close is it to 1000?

		\end{enumerate}
	
	\item Sketch the circle of radius r, centre origin, given by $x^2 + y^2 = r^2$.  Indicate all axis intersections.
		\begin{enumerate}
			\item Let a be a positive constant.  Sketch the 4 basic transformations of this graph using constant a. e.g. $f(x+a)$, $f(ax)$, etc. Indicate all axis intersections.
			
			\item For each of your transformations, identify its geometric shape and its centre.
			
--
			
		\end{enumerate}


	\item We test integration technique. Suppose that $\int{f(x)}dx = g(x)$. \\
			Determine the following, simplifying fully. (Your answers should be in terms of $g(x)$)
		\begin{enumerate}
			\item $\int (t + t^3 + t^5)dt$
			
			\item $\int 4f(x)dx$
			
			\item $\int (x^{1/2} + x^{1/3} - f(x)) dx$
			
		\end{enumerate}


	
	
	\item Let P = (3, 5) and Q = (5, 7) be pints in the Euclidean plane.
	\begin{enumerate}
		\item Find the length of the line segment PQ, and its midpoint C.
		
		\item What is the circumference of the circle having centre C that passes through both P and Q?  And its area?
		
		\item We want to find a quadratic of the form $y(x) = ax^2 + bx$ that passes through both P and Q.  So y must satisfy $y(3) = 5$ and $y(5) = 7$.  We set up the system of equations:
		$$5 = 3^2a + 3b$$
		$$7 = 5^2a + 5b$$
		
		Solve the system to find a and b and hence give the required quadratic $y(x)$.
		
		
	\end{enumerate}
	
\end{enumerate}

\end{document}