\documentclass{article}

\usepackage{amsmath}
\usepackage{amssymb}
\usepackage{graphicx}

\begin{document}
\title{Vacation Sheet 3}
\author{Ben Eills}
\maketitle

\textit{Submit neat answers to all the questions you attempt (include all the working you wish to be marked, e.g. calculations, diagrams).  Partial answers should also be handed in: these may be awarded partial credit.  \textbf{This sheet involves no calculus!}}

\begin{enumerate}
	\item We test algebraic manipulation.  Simplify the following.
		\begin{enumerate}
			\item $\sqrt{lcm(6, 4)}\sqrt{\frac{1}{2} + 8^{-\frac{2}{3}}} $ \\
			\textit{[recall that lcm(a, b) denotes the lowest common multiple of a and b]} 
			
			\item $\dfrac{1}{7 + \sqrt{7}}$
			
			\item $(7x - 9)^2 < 9$
			
			\item $(x+5)(x-6) < (x+10)(x-1)$
			
			\item $\dfrac{x^3}{10} + \dfrac{x^2}{100} + \dfrac{x}{1000}$
		\end{enumerate}
	
	\item Sketch the graphs and indicate the graph transformation(s) involved.
		\begin{enumerate}
			\item $y = \dfrac{1}{x}$ \textbf{vs.} $y = \dfrac{3}{x}$

			\item $y = (x+1)(x+2)$ \textbf{vs.} $y = (x+3)(x+4)$
			
			\item $y = x(x-1)^2$ \textbf{vs.} $y = 4(x-2)(x-3)^2$
		\end{enumerate}


	\item We test coordinate geometry.
		\begin{enumerate}
			\item Give the relationship between the gradient of a line and the angle $\theta$ it makes with the x-axis (defined to be zero if the line is horizontal), using high school trigonometry.
			
			\item We may specify a point in the plane with a minimum of two numbers (the two coordinates of the point).  We need a minimum of 6 numbers to specify a triangle in the plane (two for each of the three vertices).  Given that a line can be fully specified with any two distinct points lying on it, it is clear that a line can be specified with 4 numbers.  Is this the minimum?  If not, give a way to specify lines using less numbers.
			
			\item Given that a line passes through the origin, what is the only other datum needed to fully specify it?
			
			\item Give the relationship between:
				\begin{enumerate}
					\item the gradients of two parallel lines
					
					\item the gradients of a line and its normal
				\end{enumerate}
			
			\item Suggest a symbol to denote the gradient of a vertical line.
		\end{enumerate}


	
	
	\item Consider the simple sequence:
			$$ (a_n) = 1, 2, 3, 4, \ldots$$
			given by the rule $a_n = n$
	\begin{enumerate}
	
		\item Write down the terms $a_{17}$, $a_{23}$ and $a_{100}$
		
		\item We use the abbreviation $S_n$ to denote the sum of the first n terms. \\
				So $S_n = \sum_{i=1}^n{a_i} = a_1 + a_2 + \cdots + a_{100}$ \\
				By simple calculation, $S_5 = 1 + 2 + 3 + 4 + 5 = 15$ \\
				Similarly determine $S_6$, $S_7$ and $S_8$ \\
				Notice that these are the number of presents given to you by your true love on the 6th, 7th and 8th days of Christmas.  Owing to the large number of presents you are receiving this Christmas you are hoping to return all the presents you receive on the 12th day to the store for their cash equivalent.  You need to know how many present you will receive on the 12th day, but are unable to do the above arithmetic for this day (the numbers are too large).  We look for an easier method.
				
		\item By representing each sum $S_n$ as a triangle of dots and using the formula for the area of a triangle, determine the number of dots in each triangle of base n, and hence a formula for $S_n$ in terms of n.
		
		\item Write down the total number of presents you hope to receive on the 12th day of Christmas. \\
				Write down the value of $1 + 2 + \cdots + 100$
		\item By taking out a factor, compute the sum $3 + 6 + 9 + \cdots + 150$	
	\end{enumerate}
	
\end{enumerate}

\end{document}