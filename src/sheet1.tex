\documentclass{article}

\usepackage{amsmath}
\usepackage{amssymb}
\usepackage{graphicx}

\begin{document}
\title{Vacation Sheet 1}
\author{Ben Eills}
\maketitle
\textit{These sheets will cover a period of about 4 weeks; they are designed to familiarize you with the content of the Edexcel C1 Syllabus.  Some questions resemble those you will encounter when sitting the exam, others are written to test and extend your understanding of the topics.}


\textit{Submit neat answers to all the questions you attempt (include all the working you wish to be marked, e.g. calculations, diagrams).  Partial answers should also be handed in: these may be awarded partial credit.}

\begin{enumerate}
	\item This question tests your understanding of basic mathematical language and grammar, as is required in the Syllabus 'Preamble'.  Decide which of the following statements are true, and which are false.  Justify your answers.
		\begin{enumerate}
			\item $n$ is even $\Rightarrow 4n + 3$ is odd
			\item $p$ is prime $\Rightarrow p+1 $ is not prime
			\item There is some real number $a$ such that $x^2 + ax + a$ has exactly one root
			\item Every \textbf{surd} of the form $\frac{1}{a + b\surd 2}$ where $a$ and $b$ are integers may be written as a fraction with a rational denominator
		\end{enumerate}
	
	\item We review here the basic "rules" of calculus you have learned thus far in the course.
		\begin{enumerate}
			\item Find the derivative of a simple power $\frac{d}{dx} Ax^n$
			\item Find the derivative of a general polynomial, $\frac{d}{dx}[ A_0 + A_1 x^1 + ... + A_n x^n]$
			\item We call a point x at which $\frac{dy}{dx} = 0$ a \textbf{critical point} or a \textbf{turning point}.  Find all turning points (if indeed there are any) of the following functions using calculus and indicate where they lie on the graphs of the functions.
			\begin{enumerate}
				\item $y = x^2 - 2x - 35$
				\item $y = (x+1)(x+2)(x+3)$
				\item $y = x/4$
				\item $y = x^3$
			\end{enumerate}
		\item Do the corresponding questions to parts (a) and (b) using integration, rather than differentiation.  Find the integrals of the functions in part (c).
		\end{enumerate}
	
	\item Consider the quadratic $z = ax^2 + bx + c$, where a, b and c are constants.
		\begin{enumerate}
			\item "Complete the square" on this quadratic and deduce the quadratic formula (used to find roots of the equation $z=0$)
			\item Write down the \textbf{discriminant} of the quadratic, and sketch the graphs corresponding to the three cases:
				\begin{enumerate}
					\item $discriminant = 0$
					\item $discriminant < 0$
					\item $discriminant > 0$
				\end{enumerate}
				[You do not need to determine points of intersection, etc.]
			\item Assuming that the discriminant is positive, write down the two roots of $z = 0$
		\end{enumerate}
	
	\item \textit{[This question is more difficult.  Try it.]}
	Let $n$ be some positive integer.
		\begin{enumerate}
			\item What can you say about the integers $2n$ and $2n+1$?
			\item Let $p(x)$ and $q(x)$ be monic polynomials of degree $2n$ and $2n+1$ respectively.  \textit{(The degree of a polynomial is the highest power of x that appears.  e.g. $3x^2 + 5$ has degree 2.  A \textbf{monic} polynomial has coefficient 1 for the highest power of x. e.g. $x^3+7$ is monic, but $2x - 9$ is not monic.)}  A function is \textit{\textbf{bounded below}} if you can draw a horizontal line that its graph never goes below.  We similarly define \textit{\textbf{bounded above}}.  So, for example, the function $x^2$ is bounded below, but not above.  What can you say about the boundedness of $p(x)$ and $q(x)$?
			\item Let $r(x)$ be a polynomial that is bounded both above and below.  What can you say about $r(x)$?  Hence determine $\frac{d}{dx} r(x)$. \textit{(Your answer should be very simple!)}
		\end{enumerate}

\end{enumerate}


\end{document}