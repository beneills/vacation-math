\documentclass{article}

\usepackage{amsmath}
\usepackage{amssymb}
\usepackage{graphicx}

\begin{document}
\title{Vacation Sheet 2}
\author{Ben Eills}
\maketitle

\textit{Submit neat answers to all the questions you attempt (include all the working you wish to be marked, e.g. calculations, diagrams).  Partial answers should also be handed in: these may be awarded partial credit.}

\begin{enumerate}
	\item Again we test understanding of mathematical language.  Decide which of the following statements are true, and which are false.  Justify your answers.
		\begin{enumerate}
			\item $b^2-4ac \geq 0$ implies that $a^2+bx+c=0$ has a root
			\item $a$ and $b$ are odd $\Rightarrow a \times b$ is odd
			\item $a^2+bx+c=0$ has a root implies that $b^2-4ac > 0$
		\end{enumerate}
	
	\item We give calculus question in the examination style.
		\begin{enumerate}
			\item Let $y = (x+3)(x-5)$ \\
			Write down the roots of $y$ and the derivative $\dfrac{dy}{dx}$ of $y$ \\
			Find the gradient of the tangent lines to y at the roots, and hence the equations of both tangent lines. \\
			Do these lines meet? If so, where? \\
			Convey the above information in a brief sketch.

			\item y is a quadratic which passes through the point $(1, 18)$ and has derivative $\dfrac{dy}{dx} = 2x - 11$ \\
			Determine y using integration.
			
			\item Differentiate and integrate the following functions. \textit{(Hint: $\pi$ is nothing more than a constant, like any other number)}
			
			\begin{enumerate}
				\item $y = x^{100}$
				\item $y = 0$
				\item $y = 1 + x + x^2$
				\item $y = 1 + x + \cdots + x^{100}$
				\item $y = x^\pi$
			\end{enumerate}
\end{enumerate}


	\item This question is about finding solutions to systems of equations.  Recall that a \textbf{solution} to a list of equations is a value that makes every separate equation true; graphically, a solution is a point where the graphs of every equation intersect.  Thus a particular system may have none, one, two, etc. solutions.  Some even have infinitely many!  Try to find all solutions of the following systems algebraically.  Draw graphs if you can.
		\begin{enumerate}
			\item 
			$$ 7x + 8y = 10 $$
			$$ 2x + 3y = 5 $$
			\item
			$$ y = x^2 $$
			$$ 2y = x + 6$$
		\end{enumerate}




\item Sometimes having quadratic equations or other polynomials in the denominator can be a problem (you will see this when you begin to apply calculus to such expressions).  We use the method of \textbf{partial fractions} to \textit{decompose} the more complicated expression into the sum of simple ones.  Use the method to simplify the following.
\begin{enumerate}

\item $$ \frac{5x + 19}{(x+3)(x+5)}$$
\item $$ \frac{1}{(x+5)(x+7)}$$
\item $$ \frac{x}{x^2 - 100}$$


\end{enumerate}

\end{enumerate}

\end{document}