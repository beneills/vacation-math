\documentclass{article}

\usepackage{amsmath}
\usepackage{amssymb}
\usepackage{graphicx}

\begin{document}
\title{Vacation Sheet 7}
\author{Ben Eills}
\maketitle

\textit{Submit neat answers to all the questions you attempt (include all the working you wish to be marked, e.g. calculations, diagrams).  Partial answers should also be handed in: these may be awarded partial credit.}

\begin{enumerate}
	\item So far, you have been performing what is known as \textbf{integration in elementary terms}, or \textbf{primitive-finding}.  Although the use of the derivative is clear to you (it gives you the gradient function), the value of the integral has not been made clear. \\
	
	The integral was first formulated in terms of finding area and was only later (surprisingly) discovered to be the reverse operation of differentiation. \\
	
	Suppose that $\int f(x) dx = F(x)$, and that we want to find the area A under the graph of f(x) between two points a and b.  We define the \textbf{definite integral} as follows:
	
	$$A = \int_a^b f(x) dx = F(b) - F(a)$$
	
	
		Find the specified areas (try to draw graphs where possible).  The first example has been done for you.
	
		\begin{enumerate}
			\item \textbf{Find the area under $f(x) = (x+2)(x-1)$ between 2 and 3.} \\
			
					\textit{We find the integral of $f(x)$.}
					\begin{align*}
					F(x)		&= \int f(x) dx \\
								&= \int (x+2)(x-1) dx \\
								&= \int (x^2 + x - 2) dx \\
								&= \frac{1}{3}x^3 + \frac{1}{2}x^2 - 2x
					\end{align*}
					
					\newpage
					
					\textit{We find the area under $f(x)$ between 2 and 3.}
					
					\begin{align*}
					Area		&= \int_2^3 f(x) dx \\
								&= F(3) - F(2) \\
								&= (\frac{1}{3}3^3 + \frac{1}{2}3^2 - 2 \cdot 3) 
								- (\frac{1}{3}2^3 + \frac{1}{2}2^2 - 2 \cdot 2) \\
								&= \frac{15}{2} - \frac{2}{3} \\
								&= \frac{41}{6}						
					\end{align*}
					
			\item Find the area under $f(x) = 2x$ between 0 and 5.  Check your answer with the formula for the area of a triangle.
			
			\item Find the area of the shape bounded by $f(x) = -(x-3)(x+3)$ and the x-axis (i.e. between -3 and +3)
				
			\item Find $\int_1^9 x^{\frac{1}{2}} dx$

		\end{enumerate}
	
	\item Solve the following inequalities, giving your answers in the simplest form possible.
	
		\begin{enumerate}
			\item $x^2 + 5x - 5 < 3x + 10$
			
			\item $(x-4)(x-6) < 0$ and $2x - 7 \le 3$
			
			\item
			\begin{enumerate}
				\item $z(z+7) > 0$
				\item $z(z+7) = 0$
				\item $z(z+7) < 0$
			\end{enumerate}			
		\end{enumerate}


	\item Let $\phi(t) = t^2 - 5t - 11$ and $\theta(t) = t^2 + kt - k$
		\begin{enumerate}
			\item What are the discriminants $\Delta_{\phi}$ and $\Delta_{\theta}$ of $\phi(t)$ and $\theta(t)$, respectively?
			
			\item Give the intervals where $\phi(t)$ is positive, negative and zero.
			
			\item Similarly, do the above for $\theta(t)$ (giving your answer in terms of k).
			
			\item Write down some value of k so that $\theta(t)$ has two roots.  Give the roots of $\phi(t)$ and $\theta(t)$.
		\end{enumerate}


	
	
	\item Let P = (3, 5), Q = (9, 5) and R = (4, 7) be points in the Euclidean plane. 
	\begin{enumerate}
		\item Give the perimeter of triangle $\triangle_{PQR}$
		
		\item Give the area of triangle $\triangle_{PQR}$
		
		\item Can you find the \textbf{midpoint} of $\triangle_{PQR}$?  (This is the point where the shape would be perfectly balanced if rested on a pin)
	\end{enumerate}
	
\end{enumerate}

\end{document}